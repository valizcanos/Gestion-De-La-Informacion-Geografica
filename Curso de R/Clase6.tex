\documentclass[12pt]{beamer}
\usepackage[utf8]{inputenc}
\usepackage[T1]{fontenc}
\usepackage{lmodern}
\usepackage[spanish]{babel}
\usepackage{amsmath}
\usepackage{amsfonts}
\usepackage{amssymb}
\usepackage{graphicx}
\usetheme{AnnArbor}
\begin{document}
	\author{Victor Augusto Lizcano Sandoval}
	\title{Clase de programación en R - N$^{0}$.5}
	%\subtitle{}
	%\logo{}
	%\institute{}
	%\date{}
	%\subject{}
	%\setbeamercovered{transparent}
	%\setbeamertemplate{navigation symbols}{}
	\begin{frame}[plain]
		\maketitle
	\end{frame}
	
	\begin{frame}
		\frametitle{Gráficos en R}
		Para realizar gráficos en \textbf{R} usamos la función \textit{plot()}.
	\end{frame}

\begin{frame}
	\frametitle{Gráficos en R - Ejemplo}
	x $<-$ c(1, 2, 3, 4, 5)\\
	y $<-$ c(3, 7, 8, 9, 12)\\
	
	plot(x, y)
\end{frame}

\begin{frame}
	\frametitle{Gráficos en R - Ejemplo}
	
	\begin{figure}
		\centering
		\includegraphics[width=0.5\linewidth]{Grafica1}
		
		\label{fig:grafica1}
	\end{figure}
\end{frame}

	\begin{frame}
		\frametitle{Gráficos en R - Ejemplo: tipos}
		j =1:20\\
		k = x\\
		
		par(mfrow = c(1, 3))\\
		
		plot(j, k, type = "l", main = "type = 'l'")\\
		plot(j, k, type = "s", main = "type = 's'")\\
		plot(j, k, type = "p", main = "type = 'p'")\\
		
		par(mfrow = c(1, 1))\\
		
		plot(x, y)
	\end{frame}

	\begin{frame}
		\frametitle{Gráficos en R - Ejemplo: tipos}
		\begin{figure}
			\centering
			\includegraphics[width=0.6\linewidth]{Grafica2}
			
			\label{fig:grafica2}
		\end{figure}
	\end{frame}

	\begin{frame}
		\frametitle{Gráficos en R - Ejemplo: tipos}
			j =1:20\\
		    k = x\\
		    par(mfrow = c(1, 3))\\
		    
		    plot(j, k, type = "o", main = "type = 'o'")\\
		    plot(j, k, type = "b", main = "type = 'b'")\\
		    plot(j, k, type = "h", main = "type = 'h'")\\
		    
		    par(mfrow = c(1, 1))
	\end{frame}

	\begin{frame}
		\frametitle{Gráficos en R - Ejemplo: tipos}
		\begin{figure}
			\centering
			\includegraphics[width=0.6\linewidth]{Grafica3}
		
			\label{fig:grafica3}
		\end{figure}
	\end{frame}

	\begin{frame}
		\frametitle{Gráficos en R - Ejemplo: tipos}
		
		\begin{table}
			\begin{tabular}{cc}
				\hline
				Tipo& Descripción \\
				\hline
				l& linea \\
				
				p&puntos  \\
				
				b&puntos y lineas \\
				
				o&puntos y lineas sobrepuestas \\
				
				s&escaleras  \\
				
				h&estilo histograma  \\
				
				n&no gráfica  \\
				\hline
			\end{tabular}
		\end{table}
	\end{frame}

	\begin{frame}
		\frametitle{Gráficos en R - Parámetros}
		
		\begin{table}
			\begin{tabular}{cc}
				\hline
				Parámetro& Descripción \\
				\hline
				pch&  Tipo de puntos (va de 1 a 25)\\
				
				lwd&Ancho de linea (va de 0 a 6) \\
				
				col&  Color del gráfico \\
				
				lty &  Estilo de línea (va de 0 a 6)\\
				
				cex& Cambia el tamaño de los puntos (se maneja en fracciones)\\
				
				xlim&  Límite del eje x \\
				
				ylim & Límite del eje y  \\
				
				bg & Color de fondo  \\
				
				main & Título  \\
				
				cex.main & Tamaño  de título  \\
				
				cex.axis & Tamaño  anotaciones de los ejes\\
				
				cex.label & Tamaño  anotaciones de las etiquetas\\
				
				\hline
			\end{tabular}
		\end{table}
	\end{frame}

	\begin{frame}
		\frametitle{Gráficos en R - Parámetros}
		\begin{table}
			\begin{tabular}{cc}
				\hline
				Parámetro& Descripción \\
				\hline
				cex.sub&  Tamaño  de subtítulos\\
				fg&  Trazar color de primer plano\\
				font&  Fuente\\
				font.axis&  Fuente de anotaciones de eje\\
				font.lab&  Fuente de anotaciones de etiqueta\\
				font.main&  Fuente de titulo\\
				family&  Familia de fuente\\
				\hline
			\end{tabular}
		\end{table}
	\end{frame}


\end{document}