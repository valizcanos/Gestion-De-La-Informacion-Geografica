\documentclass[12pt]{beamer}
\usepackage[utf8]{inputenc}
\usepackage[T1]{fontenc}
\usepackage{lmodern}
\usepackage[english]{babel}
\usepackage{amsmath}
\usepackage{amsfonts}
\usepackage{amssymb}
\usepackage{graphicx}
\usetheme{AnnArbor}
\begin{document}
	\author{Victor Augusto Lizcano Sandoval}
	\title{Clase de programación en R - N$^{0}$.1}
	%\subtitle{}
	%\logo{}
	%\institute{}
	%\date{}
	%\subject{}
	%\setbeamercovered{transparent}
	%\setbeamertemplate{navigation symbols}{}
	\begin{frame}[plain]
		\maketitle
	\end{frame}
	
	\begin{frame}
		\frametitle{¿Qué es R?}
		R es un lenguaje de programación comúnmente empleado para análisis estadístico y matemático. 
	\end{frame}

	\begin{frame}
		\frametitle{"Hello World!"}

		\texttt{> Hello World!}\\
		\texttt{Error: unexpected symbol in "Hello World"}\\
		\texttt{> "Hello World!"}\\
		\texttt{[1] "Hello World!"}\\
		\texttt{> print("Hello World!")}\\
		\texttt{[1] "Hello World!"}	\\					
	\end{frame}

	\begin{frame}
		\frametitle{Función print()}
		\texttt{> print(5+5)}\\
		\texttt{[1] 10}\\
		\texttt{> print("5+5")}\\
		\texttt{[1] "5+5"}		
		
	\end{frame}

	\begin{frame}
	\frametitle{Función cat()}
	
	La función cat() se utiliza para imprimir en la pantalla o en un archivo.
	
	\texttt{> cat("Hola como estás?")}\\
	\texttt{Hola como estás?}\\
	\texttt{> cat("hola\textbackslash ncómo\textbackslash nestás?")}\\
	\texttt{hola}\\
	\texttt{cómo}\\
	\texttt{estás?}		
\end{frame}

	\begin{frame}
		\frametitle{Función cat()}
		\texttt{> cat(5+5)}\\
		\texttt{10}\\
		\texttt{> cat("5+5")}\\
		\texttt{5+5}\\
		\texttt{> cat(1:5, sep="-")}\\
		\texttt{1-2-3-4-5}\\
	\end{frame}

	\begin{frame}
		\frametitle{Comentarios \#}
		
		\texttt{>\# Este es un comentario}
	\end{frame}

	\begin{frame}
		\frametitle{Variables}
		
		Las variables son contenedores para el almacenamiento de datos.
		
	\end{frame}

	\begin{frame}
	\frametitle{Variables}
	
	Las variables son contenedores para el almacenamiento de datos.
	
\end{frame}

	\begin{frame}
	\frametitle{Variables}
		\texttt{> \#variables \\}
		\texttt{> Nombre = "María"}\\
		\texttt{> Apellido <- "Magdalena"}\\
		\texttt{> Nombre}\\
		\texttt{[1] "María"}\\
		\texttt{> Apellido}\\
		\texttt{[1] "Magdalena"}
	\end{frame}

	\begin{frame}
		\frametitle{Variables}
		
		\texttt{> x = 2}\\
		\texttt{> y = 5}\\
		\texttt{> x}\\
		\texttt{[1] 2}\\
		\texttt{> y}\\
		\texttt{[1] 5}
		
	\end{frame}


	\begin{frame}
		\frametitle{Notación de variables}
		
		1. Camel Case (contarElementos)\\
		2. Pascal Case (ContarElementos)\\
		3. Snake Case (contar\_elementos)\\
		4. Kebab Case (contar$-$elementos)
		
	\end{frame}

	\begin{frame}
	\frametitle{Reglas - declaración de variables}
	
	\begin{itemize}
		\item Una variable, siempre debe iniciar con una letra (mayúscula o minúscula) ó un guión bajo (\_).
		\item Una variable, puede contener números, solamente después de  la primer letra (siguiendo la regla anterior).
		\item No es permitido dejar un espacio en blanco a lo largo de la variable.
		\item Aunque una variable puede ser del largo que tú desees, lo recomendable es que sea una variable corta (regularmente entre 20 y 30 caracteres como máximo).
		\item No puedes utilizar palabras reservadas para la declaración de una variable.
		\item El nombre de una variable en R es case sensitive (es decir, a lo largo de tu programa debe escribirse exactamente igual).
		\item Utiliza un nombre que exprese algo del contexto en el cual la estás declarando.
	\end{itemize}
\end{frame}


	\begin{frame}
		\frametitle{Múltiples variables }
		\texttt{var1 <$-$ var2 <$-$ var3 <$-$ "Naranja"}\\
		\texttt{var1}\\
		\texttt{[1] "Naranja"}\\
		\texttt{var2}\\
		\texttt{[1] "Naranja"}\\
		\texttt{var3}\\
		\texttt{[1] "Naranja"}
	\end{frame}
	

	\begin{frame}
		\frametitle{Concatenación de elementos}
		\texttt{\#Concatenación}\\
		\texttt{Texto = "Fantástica"}\\
		\texttt{paste("Eres una persona",Texto, sep=" ")}\\
		\texttt{[1] "Eres una persona Fantástica"}
	\end{frame}

	\begin{frame}
		\frametitle{Tipos de datos}
		\texttt{\# numeric}\\
		\texttt{x <$-$ 10.5}\\
		\texttt{class(x)}\\
		\texttt{\# integer}\\
		\texttt{x <$- $1000L}\\
		\texttt{class(x)}\\
		\texttt{\# complex}\\
		\texttt{x <$-$ 9i + 3}\\
		\texttt{class(x)}\\
		\texttt{\# character/string}\\
		\texttt{x <$-$ "R is exciting"}\\
		\texttt{class(x)}\\
		\texttt{\# logical/boolean}\\
		\texttt{x <$-$ TRUE}\\
		\texttt{class(x)}		
		
	\end{frame}


	\begin{frame}
	\frametitle{Conversión de tipo}
	
	\texttt{x = 1L \# integer}\\
	\texttt{y = 2 \# numeric}\\
	\texttt{\# convert from integer to numeric:}\\
	\texttt{a = as.numeric(x)}\\
	\texttt{\# convert from numeric to integer:}\\
	\texttt{b <$-$ as.integer(y)}\\
	\texttt{\# print values of x and y}\\
	\texttt{x}\\
	\texttt{y}\\
	\texttt{\# print the class name of a and b}\\
	\texttt{class(a)}\\
	\texttt{class(b)}
	
	\end{frame}


	\begin{frame}
		\frametitle{Matemáticas}
		\texttt{10 $+$ 5 \#Suma}\\
		\texttt{10 $-$ 5 \#Resta}\\
		\texttt{max(5, 10, 15) \#Máximo}\\
		\texttt{min(5, 10, 15) \#Mínimo}\\
		\texttt{sqrt(16) \#Raíz cuadrada}\\
		\texttt{sd(1:5) \#Desviación estandar}\\
		\texttt{abs(-4.7) \#Valor absoluto}\\
		\texttt{ceiling(1.4) \#Redondear arriba}\\
		\texttt{floor(1.4) \#Redondear abajo}
	\end{frame}


	\begin{frame}
		\frametitle{Cadenas de texto}
		\texttt{str <$-$ "We are the so-called  $\backslash$"Vikings$\backslash$"  , from the north."}\\
		\texttt{cat(str)}\\
		\texttt{We are the so-called "Vikings", from the north.}\\
		\texttt{str}\\
		\texttt{[1] "We are the so-called  $\backslash$"Vikings$\backslash$" , from the north."}\\
		\texttt{nchar(str) \#Longitud de la cadena}
		
	\end{frame}

	\begin{frame}
		\frametitle{Caracteres especiales}
		
		\begin{table}
			\centering
		\begin{tabular}{cc}
			\hline
			$\backslash \backslash$  & Backslash\\
			$\backslash$n& Nueva linea\\
			$\backslash$r&  Retorno\\
			$\backslash$t&  tab\\
			$\backslash$b&  Retroceso\\
			\hline
		\end{tabular}
	\end{table}
		
	\end{frame}

	\begin{frame}
		\frametitle{Booleanos}
		
		10 > 9    \# TRUE because 10 is greater than 9 \\
		10 == 9   \# FALSE because 10 is not equal to 9 \\
		10 < 9    \# FALSE because 10 is greater than 9
		
	\end{frame}

	\begin{frame}
		\frametitle{Operadores aritmeticos}
		Suma ($+$)\\
		Resta ($-$) \\
		Multiplicación ($*$)\\
		División ($/$) \\
		Exponente ($\hat{}$)\\
		Módulo ($\%\%$) \\
		Divisón de enteros ($\%/\%$)
	\end{frame}

	\begin{frame}
		\frametitle{Operadores de asignación}
		
		my\_var <$-$ 3 \\
		
		my\_var = 3 \\
		
		my\_var $\ll$$-$ 3\\
		
		3 $-$> my\_var\\
		
		3 $-$$\gg$ my\_var		
		
	\end{frame}

	\begin{frame}
		\frametitle{Operadores de comparación}
		\begin{table}
			\centering
		\begin{tabular}{cc}
			\hline
			$==$  & Igual\\
			$!=$& No igual\\
			$>$&  Mayor que\\
			$<$& Menor que\\
			$>=$&  Mayor o igual que\\
			$<=$& Menor o igual que\\
			\hline
		\end{tabular}
	\end{table}
	
	\end{frame}

	\begin{frame}
		\frametitle{Operadores lógicos}
		
		\begin{table}
			\centering
			\begin{tabular}{cc}
				\hline
				$\&$  &Operador "y" \\
				$\&\&$& Operador "y"\\
				$|$&  Operador "o"\\
				$||$& Operador "o"\\
				$!$&  Operador "No"\\
				\hline
			\end{tabular}
		\end{table}
		
		
	\end{frame}

	\begin{frame}
		\frametitle{Operadores miscelaneos}
		
		\begin{table}
			\centering
			\begin{tabular}{cc}
				\hline
				$:$&  Crear secuencia de números\\
				$\%in\%$& Buscar un elemento dentro un vector\\
				$\%*\%$&  Multiplicar matriz\\
				\hline
			\end{tabular}
		\end{table}
		
	\end{frame}


\end{document}