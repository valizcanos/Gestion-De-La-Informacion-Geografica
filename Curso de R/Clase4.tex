\documentclass[14pt]{beamer}
\usepackage[utf8]{inputenc}
\usepackage[T1]{fontenc}
\usepackage{lmodern}
\usepackage[english]{babel}
\usepackage{amsmath}
\usepackage{amsfonts}
\usepackage{amssymb}
\usepackage{graphicx}
\usetheme{AnnArbor}
\begin{document}
	\author{Victor Augusto Lizcano Sandoval}
	\title{Clase de programación en R - N$^{0}$.4}
	%\subtitle{}
	%\logo{}
	%\institute{}
	%\date{}
	%\subject{}
	%\setbeamercovered{transparent}
	%\setbeamertemplate{navigation symbols}{}
	\begin{frame}[plain]
		\maketitle
	\end{frame}
	
	\begin{frame}
		\frametitle{Entradas de usuario}
		Para recibir alguna entrada o información de un usuario hacemos uso  la función \textit{readline()} y del operador \textit{prompt}. La función \textit{readline()} retornará un elemento de tipo carácter. Si se desea una salida en números, toca hacer la conversión.
	\end{frame}

\begin{frame}
	\frametitle{Entradas de usuario - Ejemplo}
	
	print("Hola, ¿cómo estás?")\\
	cat("$\backslash$n")\\
	Nombre = readline(prompt ="¿Cómo te llamas? ")\\
	cat("$\backslash$n")\\
	print(paste("Mucho gusto ", Nombre, " Me llamo Darth Vader"))
	cat("$\backslash$n")\\
	Edad = as.integer(readline(prompt ="¿Por cierto ... ¿cuál es tu edad ? "))\\
	cat("$\backslash$n")\\
	print(paste("Tienes ",Edad, " ?", " Wow, soy mucho mayor que tu" ))
		
\end{frame}

\begin{frame}
	\frametitle{Entradas de usuario - Taller}
	
	Un método matemático muy famoso para estimar el número pi ($\pi$) son las series infinitas de Euler (o problema de Basilea). Este método consiste en sumar los inversos cuadrados de números enteros positivos para obtener un valor equivalente a $\frac{\pi}{6}$.\\
	
	\begin{equation}
	\sum_{n=1}^{\infty} \frac{1}{n^2} = \frac{1}{1^2} + \frac{1}{2^2} + \frac{1}{3^2} + \cdots  = \frac{\pi^2}{6}
	\end{equation}
	\begin{equation}
	\pi =  \sqrt{ 6 \left( \sum_{n=1}^{\infty} \frac{1}{n^2} \right)}
	\end{equation}
	
\end{frame}

\begin{frame}
	\frametitle{Entradas de usuario - Taller}
	
	Estimar el número $\pi$ con R empleando el problema de Basilea. Para ello debe crear una función vacia y la función \textit{readline()} para pedir la extensión la longitud de la serie.
	
\end{frame}

\begin{frame}
	\frametitle{Entradas de usuario - Taller}
options(digits=15)\\
\vspace{0.5cm}

calcularPI = function(n)$\{$\\
	n = as.integer(readline(prompt="Ingrese un número entero positivo que abarque la totalidad de los valores de la sumatoria Pi: "))\\
	
	c = rep(0,n)\\
	for( i in 1:n)$\{$\\
		c[i] = 1/(i$\wedge$2)$\}$\\
	PIValor = sqrt(6$*$sum(c))\\
	return(PIValor)$\}$\\
\vspace{0.5cm}

calcularPI()

\end{frame}

\begin{frame}
	\frametitle{Función switch()}
	La función \textit{switch () } en R evalua una expresión con elementos de una lista. Si el valor evaluado de la expresión coincide con el elemento de la lista, se devuelve el valor correspondiente.
\end{frame}

\begin{frame}
	\frametitle{Función switch() - Ejemplo}
	switch(2,"Rojo","Verde", "Amarillo", "Azul") \\
	\vspace{0.5cm}
	switch("Color", "Color" = "Rojo", "Forma" = "Circulo", "Longitud" = 15)
\end{frame}

\begin{frame}
	\frametitle{Función switch() - Ejemplo}
	val1 $=$ 6  \\
	val2 $=$ 7 \\
	val3 $=$ "s"  \\ 
	resultado $=$ switch(  \\
	val3,  \\
	"s"$=$ cat("Suma =", val1 + val2),  \\
	"r"$=$ cat("Resta =", val1 - val2),  \\
	"d"$=$ cat("Division = ", val1 / val2),\\  
	"M"$=$ cat("Multiplicacion =", val1 * val2),\\
	"m"$=$ cat("Modulo =", val1 \%\% val2),\\
	"p"$=$ cat("Potencia =", val1 $\wedge$ val2)\\
	)  
	
	print(result) 
\end{frame}

\begin{frame}
	\frametitle{Vectores}
	
	Un vector es simplemente una lista de elementos que son del \textbf{mismo tipo}.\\
	\vspace{0.5cm}
	
	frutas = c("Peras", "Manzanas", "Piñas", "Bananos")\\
	ValLog = c(TRUE, FALSE,TRUE)\\
	Secuencias = seq(1,10,0.5)\\
	Numeros = c(1,3,4,6,3,2,7,8,9)
	
\end{frame}

\begin{frame}
	\frametitle{Vectores - Filtrar}
	
	Nombres = c("Ana", "Maria", "Juan", "Pedro", "Eva")\\
	
	Nombres[3]\\
	length(Nombres)
	
	
\end{frame}

\begin{frame}
	\frametitle{Listas}
	Una lista es una colección de datos ordenados y modificables de \textbf{diferentes tipos}.\\
	\vspace{0.5cm}
	frutas = list("Peras", "Manzanas", "Piñas", "Bananos")\\
	frutas[3]
\end{frame}

\begin{frame}
	\frametitle{Listas}
	Listas = list(Frutas=c("Peras", "Manzanas", "Piñas", "Bananos"), Nombres = c("Juan", "Pedro", "Ana"), Numeros = c(1:15)) \\
	\vspace{0.5cm}
	Listas[[1]][[2]] \\
	length(Listas)
\end{frame}

\begin{frame}
	\frametitle{Listas - Añadir elementos}
	Frutas = list("Pera","Manzana", "Banano")\\
	append(Frutas, "Naranja")\\
	Frutas = append(Frutas, "Mandarina", after=2)
	\vspace{0.5cm}
	
	Listas = list(Frutas=c("Peras", "Manzanas", "Piñas", "Bananos"), Nombres = c("Juan", "Pedro", "Ana"), Numeros = c(1:15))\\
	
	Listas = append(Listas[[2]], "Maria", after=2)
	
\end{frame}


\begin{frame}
	\frametitle{Listas - remover elementos}
	Frutas = list("Pera","Manzana","Mandarina", "Banano","Naranja")\\
	Frutas= Frutas[-2]\\
	Listas = list(Frutas=c("Peras", "Manzanas", "Piñas", "Bananos"), Nombres = c("Juan", "Pedro", "Ana"), Numeros = c(1:15))\\
	Listas = Listas[[3]][-10]
\end{frame}

\begin{frame}
	\frametitle{Listas - remover elementos}
	Frutas = list("Pera","Manzana","Mandarina", "Banano","Naranja")\\
	Frutas= Frutas[-2]\\
	Listas = list(Frutas=c("Peras", "Manzanas", "Piñas", "Bananos"), Nombres = c("Juan", "Pedro", "Ana"), Numeros = c(1:15))\\
	Listas = Listas[[3]][-10]
\end{frame}

\begin{frame}
	\frametitle{Listas - rangos}
	Listas = list(Frutas=c("Peras", "Manzanas", "Piñas", "Bananos"), Nombres = c("Juan", "Pedro", "Ana"), Numeros = c(1:15))\\
	(Listas)[[2]][1:3]
\end{frame}

\begin{frame}
	\frametitle{Listas - concatenar}
	Listas = list(Frutas=c("Peras", "Manzanas", "Piñas", "Bananos"), Nombres = c("Juan", "Pedro", "Ana"), Numeros = c(1:15))\\
	Listas2 = list(Animales=c("Perro","Gato","Conejo"))\\
	NuevaLista = c(Listas, Listas2)
\end{frame}

\begin{frame}
	\frametitle{Listas - blucles}
	Listas = list(Frutas=c("Peras", "Manzanas", "Piñas", "Bananos"), Nombres = c("Juan", "Pedro", "Ana"), Numeros = c(1:15))\\

	\vspace{0.5cm}
	for (x in Listas[[2]]) $\{$\\
		print(x)\\
	$\}$
\end{frame}

\begin{frame}
	\frametitle{Matrices}
	Una matriz es un conjunto de datos (\textbf{de un solo tipo}) bidimensionales con columnas y filas.
\end{frame}

\begin{frame}
	\frametitle{Matrices-Ejemplo}
	LaMatriz = matrix(1:12, ncol=3, nrow=4, byrow= TRUE)
\end{frame}

\begin{frame}
	\frametitle{Matrices-Filtrar}
	LaMatriz[2,2] $\#$Elementos\\
	LaMatriz[2,]  $\#$ Filas\\
	LaMatriz[,2] $\#$ Columnas\\
	LaMatriz[c(1,3),] $\#$ Filas especificas\\
	LaMatriz[,c(1,3)] $\#$ Columnas especificas
	
\end{frame}

\begin{frame}
	\frametitle{Matrices-agregar filas y columnas}
	Lamatriz = cbind(Lamatriz, c(4,7,10,13)) $\#$Agregar columna\\
	Lamatriz = rbind(Lamatriz, c(13,14,15,16)) $\#$Agregar fila\\
	
\end{frame}

\begin{frame}
	\frametitle{Matrices-remover filas y columnas}
	Lamatriz = LaMatriz[-c(1),-c(1)] $\#$Remover fila 1  y columna 1\\
	
\end{frame}

\begin{frame}
	\frametitle{Matrices-Chequear la existencia de algún elemento}
	8\%in\%LaMatriz
\end{frame}

\begin{frame}
	\frametitle{Matrices-dimensiones y longitudes de elementos}
	dim(LaMatriz)\\
	nrow(LaMatriz)\\
	ncol(LaMatriz)\\
	length(LaMatriz)
\end{frame}

\begin{frame}
	\frametitle{Matrices - bucles en matrices}

 for (filas in 1:nrow(LaMatriz)) $\{$\\
	for (columnas in 1:ncol(LaMatriz)) $\{$\\
		print(LaMatriz[filas, columnas])\\
	$\}$\\
$\}$

\end{frame}

\begin{frame}
	\frametitle{Matrices - Operaciones (Transpuesta)}
	
	A = matrix(c(10, 8, 5, 12), ncol = 2, byrow = TRUE)\\
	B = matrix(c(5, 3,15, 6), ncol = 2, byrow = TRUE)\\
	t(A)\\
	t(B)
\end{frame}

\begin{frame}
	\frametitle{Matrices - Operaciones (suma-resta)}	
	A+B\\
	A-B
\end{frame}

\begin{frame}
	\frametitle{Matrices - Operaciones (Multiplicación)}	
	2*A $\#$ Multiplicar por un escalar\\
	A*B  $\#$ Multiplicar elemento a elemento\\
	A\%*\%B $\#$ Multiplicación matricial \\
	crossprod(A, B)  $\#$ Producto cruzado  t(A)\%*\% B \\
	tcrossprod(A, B) $\#$ Producto cruzado  A\%*\%t(B) \\
\end{frame}

\begin{frame}
	\frametitle{Matrices - Operaciones (potencia)}
	
	Para ello instalamos la siguiente libreria:\\
	install.packages("expm", dependencies=TRUE)\\
	library(expm)\\
	A \%$\wedge$\% 2
	
\end{frame}

\begin{frame}
	\frametitle{Matrices - Operaciones (determinante)}
	
	det(A)\\
	det(B)
	
\end{frame}

\begin{frame}
	\frametitle{Matrices - Operaciones (inversa)}
	
	solve(A)\\
	solve(B)\\
	solve(A, B) $\#$ Resuelve  un sistema de ecuaciones A\%*\% X =B
	
\end{frame}

\begin{frame}
	\frametitle{Matrices - Operaciones (diagonal)}
	
	diag(A)\\
	diag(B)\\
	diag(4) $\#$ Genera una matriz identidad
	
\end{frame}

\begin{frame}
	\frametitle{Matrices - Operaciones (autovalores)}
	
	eigen(A)$\$$values\\
	eigen(B)$\$$values\\
	eigen(A)$\$$vectores\\
	eigen(B)$\$$vectores
	
\end{frame}

\begin{frame}
	\frametitle{Arreglos}
	Son matrices con más de dos dimensiones.\\
	Arreglo = c(1:24)\\
	Arreglo = array(Arreglo, dim = c(4, 3, 2))\\
	Arreglo[,,2]
	
\end{frame}

\begin{frame}
	\frametitle{Dataframes}
	Son  datos que se muestran en formato de tabla. A diferencia de las matrices, sus columnas (o campos), pueden contener datos de diferente tipo.
	
\end{frame}

\begin{frame}
	\frametitle{Dataframes - ejemplo}
	Df = data.frame(Genero = c("Hombre","Mujer","Mujer","Mujer", "Hombre"), Edad = c(25,28,26,22,27), Ciudad= c("Cali","Bogotá","Medellín","Barranquilla","Bucaramanga") )
	
\end{frame}

\begin{frame}
	\frametitle{Dataframes - ejemplo (resumen estadístico y estructura)}
	
	summary(Df)\\
	str(Df)
	
\end{frame}

\begin{frame}
	\frametitle{Dataframes - ejemplo (Filtros por columna)}
	
	Df[[1]]\\
	Df[1]\\
	Df[["Genero"]]\\
	Df["Genero"] \\
	Df$\$$Genero\\
	Los demás filtros y funciones empleadas en matrices aplican para los dataframes.
	
\end{frame}

\begin{frame}
	\frametitle{Dataframes - ejemplo (Operaciones)}
	
	Df$\$$EdadPor2 = Df$\$$*2
	
\end{frame}

\begin{frame}
	\frametitle{Dataframes - ejemplo (Nombre Columna)}
	
	colnames(Df)[4] = c("Edad duplicada")
	
\end{frame}

\begin{frame}
	\frametitle{Dataframes - ejemplo (Nombre fila)}
	
	rownames(Df) = c("Dato 1", "Dato 2","Dato 3","Dato 4","Dato 5")
	
\end{frame}

\begin{frame}
	\frametitle{Dataframes - ejemplo (Cambiar valores)}
	
	Df$\$$Edad[Df$\$$Edad==28]=25 \\
	Df$\$$Edad[1]=29
	
\end{frame}

\begin{frame}
	\frametitle{Factores}
	Los factores se utilizan para categorizar datos.
	
\end{frame}

\begin{frame}
	\frametitle{Factores-Ejempo}
	GeneroMusical <- factor(c("Jazz", "Rock", "Clasica", "Clasica", "Pop", "Jazz", "Rock", "Jazz"))\\
	levels(GeneroMusical)
	
\end{frame}

\begin{frame}
	\frametitle{Fechas-Ejempo}
	navidad=as.Date("2013-12-25")\\
	navidad=as.Date("25/12/2013",format="\%d/\%m/\%Y")\\
	navidad=as.Date("25-dec-13",format="\%d-\%b-\%y")\\
	navidad=as.Date("25 December 2013",format="\%d \%B \%Y")
	
\end{frame}

\begin{frame}
	\frametitle{Fechas-Ejempo}
	
	\begin{table}
		\caption{Formatos de fecha en R}
		\begin{tabular}{cc}
			\hline
			Simbolo& Significado \\
			\hline
			\%d &  día (numérico, de 0 a 31)\\
			
			\%a &  día de la semana abreviado a tres letras\\
			
			\%A & día de la semana (nombre completo)\\
			
			\%m &  mes (numérico de 0 a 12)\\
			
			\%b &  mes (nombre abreviado a tres letras)\\
			
			\%B &  mes (nombre completo)\\
			
			\%y &  año (con dos dígitos)\\
			
			\%Y &  año (con cuatro dígitos)\\
			\hline
		\end{tabular}
	\end{table}
	
\end{frame}

\begin{frame}
	\frametitle{Fechas-Ejempo}
	Fecha1 = seq(as.Date("2020-01-01"), as.Date("2020-12-31"), by="days")\\
	Fecha1 = seq(as.Date("2020-01-01"), as.Date("2020-12-31"), by="weeks")\\
	Fecha1 = seq(as.Date("2020-01-01"), as.Date("2020-12-31"), by="2 weeks")\\
	Fecha2 = seq(as.Date("2020-01-01"), as.Date("2020-12-31"), by="months")\\
	Fecha3 = seq(as.Date("2020-01-01"), as.Date("2020-12-31"), by="quarters")\\
	Fecha4 = seq(as.Date("2000-01-01"), as.Date("2020-12-31"), by="years")
	
\end{frame}

\begin{frame}
	\frametitle{Fechas-Ejempo}
	format(Fecha1, "\%Y")\\
	format(Fecha1, "\%y")\\
	format(Fecha1, "\%m")\\
	format(Fecha1, "\%d")\\
	Sys.Date()
	
\end{frame}



\begin{frame}
	\frametitle{Fechas-Ejempo}
	dia1=as.Date("25/12/2012",format="\%d/\%m/\%Y")
	dia2=as.Date("20/1/2013",format="\%d/\%m/\%Y")
	dia3=as.Date("25/12/2013",format="\%d/\%m/\%Y")
	dia3-dia1\\
	dia3-dia2\\
	dia2-dia1	
	
\end{frame}

\begin{frame}
	\frametitle{Fechas-Ejempo}
	difftime(dia3, dia1, units = "weeks")\\
	difftime(dia3, dia1, units = "days")\\
	difftime(dia3, dia1, units = "hours")\\
	difftime(dia3, dia1, units = "mins")\\
	difftime(dia3, dia1, units = "secs")\\
	dia3+10\\
	dia3+42
\end{frame}

\begin{frame}
	\frametitle{Fechas-Ejempo}
	dias=as.Date(c("1/10/2005","2/2/2006",\\
	"3/4/2006","6/8/2006"),format="\%d/\%m/\%Y")\\
	diff(dias)\\
	Calcula la diferencia, en días, entre los términos sucesivos de un vector de fechas 
\end{frame}

\begin{frame}
	\frametitle{Fechas-Ejempo}
	seq(dia1,dia3,length=10)\\
	seq(dia1,dia3,by=15)\\
\end{frame}

\begin{frame}
	\frametitle{Fechas-Ejempo}
	FechaHora = as.POSIXct("01/10/1983 22:10:00",format="\%d/\%m/\%Y \%H:\%M:\%S")\\
	
	FechaHora = as.POSIXct("01/10/1983 22:10:00",format="\%d/\%m/\%Y \%H:\%M:\%S", tz="PDT")\\
	Sys.time()
	
\end{frame}


\begin{frame}
	\frametitle{Objetos temporales-Ejempo}
	miST <- ts(c(1:72), start=c(2009, 1), end=c(2014, 12), frequency=12)\\
	
	en \textit{frequency} 12 son meses, 4 son trimestres, 1 son años.
	
\end{frame}

\begin{frame}
	\frametitle{Objetos temporales-Ejempo}
	miST2 <- stl(miST, s.window="period")\\
	
	miST3 <- window(miST, start=c(2014, 6), end=c(2014, 12))
	
	
\end{frame}

\end{document}