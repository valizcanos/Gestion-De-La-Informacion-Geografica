\documentclass[12pt]{beamer}
\usepackage[utf8]{inputenc}
\usepackage[T1]{fontenc}
\usepackage{lmodern}
\usepackage{amsmath}
\usepackage{amsfonts}
\usepackage{amssymb}
\usepackage{graphicx}
\usetheme{AnnArbor}
\begin{document}
	\author{Victor Augusto Lizcano Sandoval}
	\title{Clase de programación en R - N$^{0}$.5}
	%\subtitle{}
	%\logo{}
	%\institute{}
	%\date{}
	%\subject{}
	%\setbeamercovered{transparent}
	%\setbeamertemplate{navigation symbols}{}
	\begin{frame}[plain]
		\maketitle
	\end{frame}
	
	\begin{frame}
		\frametitle{Funciones apply}
		Las funciones apply son  usadas para aplicar funciones a elementos de una estructura de datos (matrices, data frames, arrays y listas).
	\end{frame}

\begin{frame}
	\frametitle{Función apply}
	\framesubtitle{Para matrices, arreglos y dataframes}
	df <- data.frame(x = 1:4, y = 5:8, z = 10:13)\\
	apply(X = df, MARGIN = 1, FUN = sum)\\
	MARGIN = 1 aplica para filas\\
	MARGIN = 2 aplica para columnas
\end{frame}

\begin{frame}
	\frametitle{Función lapply}
	\framesubtitle{Para listas. Recibe una lista y devuelve una lista}
	A <- matrix(1:9,nrow = 3, ncol = 3)\\
	B <- matrix(11:19,nrow = 3, ncol = 3)\\
	C <- matrix(21:29,nrow = 3, ncol = 3)\\
	miLista <- list(A,B,C)\\
	lapply(miLlista,"$[$",1,1)
\end{frame}

\begin{frame}
	\frametitle{Función sapply}
	\framesubtitle{Para listas. Recibe una lista y devuelve un vector}
	sapply(miLista,"$[$",1,1)
\end{frame}

\begin{frame}
	\frametitle{Función tapply}
	\framesubtitle{Para factores}
	x <- 1:20\\
	y <- factor(rep(letters[1:5], each = 4))\\
	tapply(x, y, sum)
\end{frame}


\begin{frame}
	\frametitle{Función mapply}
	\framesubtitle{Para matrices. Devuelve una lista o un vector}
	mapply(sum, 1:5, 1:5, 1:5) $\#$ Suma el primer elemento de cada vector, después el segundo y luego el tercero\\
	mapply(rep, 1:4, 4:1) $\#$ Repite cada elemento del primer vector el número de veces que indique el segundo vector.
\end{frame}

\begin{frame}
	\frametitle{Función vapply}
	\framesubtitle{Para listas. Devuelve un vector}
	x <- list(A = 1, B = 1:3, C = 1:7)\\
	vapply(x, FUN = length, FUN.VALUE = 0L)
\end{frame}

\begin{frame}
	\frametitle{Librería reshape}
	\framesubtitle{Para transformar la estrucutra de los datos}
	library(reshape)\\
	str(mtcars)\\
	mtcars2 = melt(mtcars,id=c("mpg","cyl"))\\
	cast(mtcars2, mpg~variable, mean)\\
	
\end{frame}

\begin{frame}
	\frametitle{Librería readxl}
	\framesubtitle{Para leer datos de Excel}
	library(readxl)\\
	misDatos <- read$\_$excel("mi$\_$archivo.xlsx", sheet = "Hoja 1")
	
\end{frame}

\begin{frame}
	\frametitle{Texto plano}
	\framesubtitle{Para leer datos de texto}
	
	misDatos <- read.csv("mi$\_$archivo.csv", header = TRUE, na.strings = "NA", dec=".",sep=",",stringsAsFactors = TRUE)\\
	misDatos <- read.csv2(file.choose(), header = TRUE, na.strings ="NA", dec=",",sep=";",stringsAsFactors = TRUE)\\
	
	misDatos <- read.table(mi$\_$archivo.csv, header = TRUE, na.strings ="NA", dec=".",sep="\textbackslash t",stringsAsFactors = TRUE)
	
\end{frame}

\begin{frame}
	\frametitle{Valores ausentes}
	\framesubtitle{ NA (not available)-NaN (not a number)}
	
	y <- c(1,2,3,NA)\\
	is.na(y) $\#$  Existencia de datos ausentes\\
	mean(y, na.rm=TRUE) $\#$ Saca el promedio sin tener en cuenta el dato ausente\\
	MisDatos[!complete.cases(MisDatos),]  $\#$ Lista las filas de datos que contienen valores ausentes\\
	DatosSiNA <- na.omit(MisDatos)  $\#$  Crea un conjunto de datos sin valores ausentes\\
	colSums(is.na(MiDataFrame)) $\#$ Suma datos ausentes por columnas
	
\end{frame}

\begin{frame}
	\frametitle{Función which}
	\framesubtitle{Devuelve la posición o el índice del valor que satisface la condición dada}
	
	UnVector<- c(5,4,3,2,1)
	which(UnVector==2)
	
\end{frame}

\begin{frame}
	\frametitle{Ordenar datos}
	\framesubtitle{Funciones order() y sort()}
	str(iris)\\
	sort(iris$\$$Sepal.Length)\\
	sort(iris$\$$Sepal.Length, decreasing=TRUE)\\
	order(iris$\$$Sepal.Length)\\
	order(iris$\$$Sepal.Length, decreasing=TRUE)
	
\end{frame}

\begin{frame}
	\frametitle{Data cheat sheets}
	\framesubtitle{Resumen para el manejo de algunas librerías}
		
		Las \textit{sata cheat sheets} facilitan el uso de algunos  paquetes. Buscar en  \url{https://www.rstudio.com/resources/cheatsheets/} y descargar la \textit{cheat sheet} de la librería \textit{dplyr}.
	
\end{frame}


\end{document}